\section{Newton-Euler Modelling}
Okay, your task is to model a system using Newton-Euler. Interesting, in my opinion the hardest of the three methods. But, let's get started.\newline
The Newton-Euler equations referenced to the center of mass in an inertial frame are given by:
\begin{equation}
    \mathbf{F}_{bc} = m \mathbf{a}_c
\end{equation}
\begin{subequations}
    \begin{equation}
        \mathbf{T}_{bc} = \mathbf{M}_{b/c} \cdot \boldsymbol{\alpha}_{ib} + \boldsymbol{\omega}_{ib} \times \left( \mathbf{M}_{b/c} \cdot \boldsymbol{\omega}_{ib} \right)
    \end{equation}
    \begin{equation}
        \boldsymbol{\tau} = \mathbf{J}\dot{\boldsymbol{\omega}} + \boldsymbol{\omega} \times \mathbf{J}\boldsymbol{\omega}
    \end{equation}
\end{subequations}
\textbf{More complicated equations: } \newline
We use this when the forces and torques are not applied at the center of mass. The equations are given by:
The total torque acting in a arbitrary point P is given by:
\begin{equation}
    \vec{F}_{bo} = m \vec{a}_c
\end{equation}
\begin{subequations}
    \begin{equation}
        \mathbf{T}_{p} = \vec{r}_{c/p} \times m \vec{a}_c \cdot \mathbf{M}_{b/c} \cdot \boldsymbol{\alpha}_{ib} + \boldsymbol{\omega}_{ib} \times \left( \mathbf{M}_{b/c} \cdot \boldsymbol{\omega}_{ib} \right)
    \end{equation}
    \begin{equation}
        \boldsymbol{\tau} = \mathbf{J}\dot{\boldsymbol{\omega}} + \boldsymbol{\omega} \times \mathbf{J}\boldsymbol{\omega} + \mathbf{r} \times m \mathbf{a}
    \end{equation}
\end{subequations}
\textbf{Even more complicated equations: } \newline
If we apply newton euler equations in a non inertial frame, we get the following equations:
Non inertial frame means that the frame is accelerating. The equations are given by:
(børge wont do us like this though??)(this comes from setting up the equations of motion in a non inertial frame and then transforming them to the inertial frame with a rotation matrix)
\begin{equation}
    m \dot{\vec{v}}_c^a + m \vec{\omega}_{ai}^a \times \vec{v}_c^a = \vec{F}_{ac}^{a}
\end{equation}
In these equations:
\begin{itemize}
    \item $\vec{r}_{p/c}$ is the vector from the point P to the center of mass.
    \item $\mathbf{F}_{bc}$: Force acting on the body at the center of mass
    \item $m$: Total mass of the body
    \item $\mathbf{a}_c$: Acceleration of the center of mass
    \item $\mathbf{T}_{bc}$: Torque acting on the body at the center of mass
    \item $\mathbf{M}_{b/c}$: Inertia matrix with respect to the center of mass
    \item $\boldsymbol{\alpha}_{ib}$: Angular acceleration of the body relative to the inertial frame
    \item $\boldsymbol{\omega}_{ib}$: Angular velocity of the body relative to the inertial frame
\end{itemize}