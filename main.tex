\documentclass[11pt, a4paper, USenglish]{article} % change ``USenglish'' to ``norsk'' if applicable.

\usepackage{graphicx} % Provides the \includegraphics command.
\usepackage{hyperref} % Provides clickable links. Always load last, but before cleveref

\usepackage{amsmath}
\usepackage{amsfonts}
\usepackage{cancel}

\usepackage{svg}
\usepackage{adjustbox}
\usepackage{pdfpages}

\usepackage{float}
\usepackage{placeins} % Include the placeins package at the beginning of your document
\usepackage{subcaption} % Include the subcaption package at the beginning of your document
\usepackage{caption} % Include the caption package at the beginning of your document
\usepackage{tikz}
\usetikzlibrary{arrows,positioning}
\usepackage{circuitikz} % for circuit diagrams

\usepackage{listings} % Include the listings package
\usepackage{xcolor} % for setting colors
\usepackage[top=0.5in, left=0.5in, right=0.5in]{geometry} % for setting the margin
\usepackage{titlesec}
\titlespacing*{\section}{0.1pt}{0.1pt}{\parskip}

% Define the MATLAB style
\lstdefinestyle{MATLAB}{
    language=Matlab, % Use the MATLAB language
    frame=single, % Add a frame around the code
    breaklines=true, % Allow line breaks
    postbreak=\mbox{\textcolor{red}{$\hookrightarrow$}\space}, % Indicate line breaks with a red arrow
    numbers=left, % Line numbers on the left
    numberstyle=\tiny\color{gray}, % Line numbers are tiny and gray
    commentstyle=\color{green}, % Comments are green
    stringstyle=\color{purple}, % Strings are purple
    keywordstyle=\color{blue}, % Keywords are blue
    basicstyle=\ttfamily % Use the typewriter font
}

%% To easily include matlab code, I would recommend to have a look at this answer https://tex.stackexchange.com/a/158816 which has nice examples.

%% It uses: https://ctan.org/pkg/matlab-prettifier

\begin{document}

% Titlepage
\title{TTK4130 Modelling and simulation}
\author{Gutta på MS 23/24}
\date{\today}
\begin{titlepage}
    \maketitle
    \begin{center}
    \includegraphics[scale=1, trim={7cm 12.5cm 7cm 9.4cm}, clip]{figures/wagon.pdf}
    \captionof*{figure}{\textit{Wagon with inverted pendulum}}
    \vfill % Add vertical space that expands to fill the available space
    \includegraphics[scale=0.4]{figures/itk_ntnu}\\
    Department of Engineering Cybernetics
    \end{center}
    \thispagestyle{empty}
\end{titlepage}

%Abstracts
\newpage
\thispagestyle{empty}
\vspace*{\fill}
\begin{abstract}
    \centering
    I want to thank everyone who contributed to this document. By that I mostly mean the boys from Fisk 22/23. This is mostly their content, but added some things that is new for this years cirriculum. This is also a very messy document. This is because we never estahblished a common goal, and also because the one from the Fiskeboys were very very good. I also want to say, that you should not rely on anything from this little compendium. We take no reponsibility. \\
    I hope we all get a good grade tomorrow, thank you for using us.
\end{abstract}
\vspace*{\fill}

% TOC
\newpage
\tableofcontents
\thispagestyle{empty} % Avoid page numbering on the table of contents.


% Main content
\newpage
\setcounter{page}{1}
\section{Kinematics}
\subsection{Vector notation}
There are two ways of expressing vectors:
\begin{subequations}
\begin{align}
    \mathbf{r}^i &= \begin{bmatrix} a \\ b \\ c \end{bmatrix} \\
    \vec{r}_{a/b} &= a\vec{i}_i + b\vec{j}_i + c\vec{k}_i
\end{align}
\end{subequations}
Where $i$ is the frame of reference, and subscript $a/b$ denotes from point b to point a. If the vectors are velocities, subscript $a/b$ denotes the velocity of point a relative to point b.
It is important to be consistent in the notation. Never do arithmetic operations on vectors expressed in different frames. This means: \newline
\begin{subequations}
\begin{align}
    \bcancel{\cancel{\vec{r}^i}}\\
    \bcancel{\cancel{\mathbf{v}^i+\mathbf{u}^a}}\\
    \bcancel{\cancel{\mathbf{v}^i \times \mathbf{u}^a}}
\end{align}
\end{subequations}

\subsection{Skew matrix notation}
If you have $\mathbf{u} = \begin{bmatrix}u_1 & u_2 & u_3\end{bmatrix}^\top$ and $\mathbf{v} = \begin{bmatrix}v_1 & v_2 & v_3\end{bmatrix}^\top$:
\begin{subequations}
    \begin{align}
    \mathbf{u}^\times = \begin{bmatrix}
                    0 & -u_3 & u_2 \\
                    u_3 & 0 & -u_1 \\
                    -u_2 & u_1 & 0
                  \end{bmatrix} \\
    \mathbf{u}^\times \mathbf{v} = \mathbf{u} \times \mathbf{v} \\
    \mathbf{u}^\times \mathbf{u} = 0 \\
    \left(\mathbf{u}^\times\right)^\times = -\mathbf{u}^\times \\
    \det{(\mathbf{u}^\times)} = 0  
    \end{align}
\end{subequations}
The cross product of two vectors can be calculated by finding the determinant of this matrix:
\begin{equation}
    \mathbf{u} \times \mathbf{v} = \det{\begin{bmatrix}
                    \vec{i} & \vec{j} & \vec{k} \\
                    u_1 & u_2 & u_3 \\
                    v_1 & v_2 & v_3
                  \end{bmatrix}}
\end{equation}

\subsection{Rotation matrices}
The rotation matrices are defined on each axis as:
\begin{subequations}
    \begin{align}
    \mathbf{R}_x(\theta) &= \begin{bmatrix} 1 & 0 & 0 \\ 0 & \cos(\theta) & -\sin(\theta) \\ 0 & \sin(\theta) & \cos(\theta) \end{bmatrix} \\
    \mathbf{R}_y(\phi) &= \begin{bmatrix} \cos(\phi) & 0 & \sin(\phi) \\ 0 & 1 & 0 \\ -\sin(\phi) & 0 & \cos(\phi) \end{bmatrix} \\
    \mathbf{R}_z(\psi) &= \begin{bmatrix} \cos(\psi) & -\sin(\psi) & 0 \\ \sin(\psi) & \cos(\psi) & 0 \\ 0 & 0 & 1 \end{bmatrix}
    \end{align}
\end{subequations}
Now lets say frame $a$ relative to frame $i$ is rotated by $\theta$ about the $x$-axis, $\phi$ about the $y$-axis, and $\psi$ about the $z$-axis. The rotation matrix from frame $a$ to frame $i$ is then:
\begin{equation}
    \mathbf{R}_i^a = \mathbf{R}_z(\psi)\mathbf{R}_y(\phi)\mathbf{R}_x(\theta)
\end{equation}
Which means that $\mathbf{R}_i^a$ is called the \textit{rotation matrix} from frame $a$ to frame $i$. \newline
\begin{figure}[H]
    \centering
    \includegraphics[scale=1, trim={7cm 3.5cm 7cm 16.5cm}, clip]{figures/arbitrary_axis_rotation.pdf}
    \caption{Rotation about arbitrary axis}
    \label{fig:arbitrary_axis_rotation}
\end{figure}
Now, if we have a rotation about an arbitrary axis $\mathbf{v}$, we can use the equation:
\begin{equation}
    \mathbf{R}_{\alpha,\vec{v}} = \cos(\alpha)\mathbf{I} + \mathbf{v}^\times\sin(\alpha) + \mathbf{vv}^\top(1-\cos(\alpha)) = \mathbf{R}_b^a(\alpha)
\end{equation}
\subsubsection{Properties of rotation matrices}
A couple of maybe interesting properties of rotation matrices:
\begin{subequations}
    \begin{align}
        \mathbf{v}^b=\mathbf{R}_a^b\mathbf{v}^a \\
        \mathbf{v}^a=\mathbf{R}_b^a\mathbf{v}^b \\
        \mathbf{R}_a^b\mathbf{R}_b^a = \mathbf{I} \\
        \mathbf{R}_a^b = \left(\mathbf{R}_b^a\right)^{-1} = \left(\mathbf{R}_b^a\right)^\top \\
        \dot{\mathbf{R}}_b^a = \left(\mathbf{\omega}_{ab}^a\right)^\times\mathbf{R}_b^a = \mathbf{R}_b^a\left(\mathbf{\omega}_{ab}^b\right)^\times \\
        \left(\vec{\omega}_{ab}^a\right)^\times = \dot{\mathbf{R}}_b^a\left(\mathbf{R}_b^a\right)^\top \\
        \vec{\omega}_{ad} = \vec{\omega}_{ab} + \vec{\omega}_{bc} + \vec{\omega}_{cd}
    \end{align}
\end{subequations}
Where the vector $\vec{\omega}_{ab}^a$ is the angular velocity vector from frame \textit{b} relative to frame \textit{b} with respect to frame \textit{a}.
\subsection{Other stuff that might be usefull}
\textbf{Linear momentum} does not depend upn its point of reference:
\begin{subequations}
    \begin{align}
        \vec{p} = m\vec{v} \\
        \dot{\vec{p}} = m\vec{a} = \vec{F}
    \end{align}
\end{subequations}
\textbf{Angular momentum} of point \textit{p} with respect t origin \textit{o}, where $\vec{r}_{p/o}$ is the position of \textit{p} and $\vec{p}$ is the linear momentum. (Angular momentum depends upon its point of reference):
\begin{subequations}
    \begin{align}
        \vec{h}_{p/o} = \vec{r}_{p/o} \times \vec{p} \\
        \dot{\vec{h}}_{p/o} = \vec{r}_{p/o} \times \dot{\vec{p}} = \vec{T}
    \end{align}
\end{subequations}
\section{Newton-Euler Modelling}
Okay, your task is to model a system using Newton-Euler. Interesting, in my opinion the hardest of the three methods. But, let's get started.\newline
The Newton-Euler equations referenced to the center of mass in an inertial frame are given by:
\begin{equation}
    \mathbf{F}_{bc} = m \mathbf{a}_c
\end{equation}
\begin{equation}
    \mathbf{T}_{bc} = \mathbf{M}_{b/c} \cdot \boldsymbol{\alpha}_{ib} + \boldsymbol{\omega}_{ib} \times \left( \mathbf{M}_{b/c} \cdot \boldsymbol{\omega}_{ib} \right)
\end{equation}
\textbf{More complicated equations: } \newline
We use this when the forces and torques are not applied at the center of mass. The equations are given by:
The total torque acting in a arbitrary point P is given by:
\begin{equation}
    \vec{F}_{bo} = m \vec{a}_c
\end{equation}
\begin{equation}
    \mathbf{T}_{p} = \vec{r}_{c/p} \times m \vec{a}_c \cdot \mathbf{M}_{b/c} \cdot \boldsymbol{\alpha}_{ib} + \boldsymbol{\omega}_{ib} \times \left( \mathbf{M}_{b/c} \cdot \boldsymbol{\omega}_{ib} \right)
\end{equation}   
\textbf{Even more complicated equations: } \newline
If we apply newton euler equations in a non inertial frame, we get the following equations:
Non inertial frame means that the frame is accelerating. The equations are given by:
(børge wont do us like this though??)(this comes from setting up the equations of motion in a non inertial frame and then transforming them to the inertial frame with a rotation matrix)
\begin{equation}
    m \dot{\vec{v}}_c^a + m \vec{\omega}_ai^a \times \vec{v}_c^a = \vec{F}_{ac}^{a}
\end{equation}

\section{Lagrange modelling}
\subsection{Generalized coordinates}
The generalized coordinates are $\mathbf{q} \in \mathbb{R}^n$, where $n \geq \text{DOF}$. DOF is degrees of freedom, and is the number of independent coordinates needed to describe the configuration of a system. The generalized coordinates are not unique, and can be chosen in many ways. The choice of generalized coordinates is important, as it can simplify the equations of motion. The generalized velocities $\dot{\mathbf{q}}$ are the time derivatives of the generalized coordinates. The generalized accelerations $\ddot{\mathbf{q}}$ are the time derivatives of the generalized velocities. The generalized forces $\mathbf{Q}$ are the forces that act on the system.
\subsection{Definition of the lagrangian $\mathcal{L}$}
\begin{subequations}
\begin{align}
    \mathcal{L} &= T - V - \mathbf{z}^\top\mathbf{c}\\
    \mathcal{L}(\mathbf{q}, \dot{\mathbf{q}}) &= T(\mathbf{q}, \dot{\mathbf{q}}) - V(\mathbf{q}) - \mathbf{z}^\top \mathbf{c(q)}
    \label{eq:Euler_Lagrange}
\end{align}
\end{subequations}
Where $T$ is the kinetic energy, $V$ is the potential energy, $\mathbf{q}$ is the generalized coordinates, $\dot{\mathbf{q}}$ is the generalized velocities, $z \in \mathbb{R}^n $ is the lagrangian multiplier and $n$ is the number of constraints. It is also valid to use $+\mathbf{z}^\top\mathbf{c}$. Equation \eqref{eq:Euler_Lagrange} \newline
\begin{subequations}
\begin{align}
    T = \frac{1}{2}m\dot{\mathbf{p}}^\top\dot{\mathbf{p}} + \frac{1}{2}J\dot{\theta}^2 \\
    \mathbf{\dot{p}} = \frac{\partial \mathbf{p}}{\partial \mathbf{q}}\dot{\mathbf{q}} \\
    T = \frac{1}{2}mv^\top v + \frac{1}{2}J\omega^\top \omega \\
    V = \underbrace{mgh}_{\text{gravity}}  + \underbrace{\frac{1}{2}kx^2}_{\text{potential energy in spring}}  \\
    T = \frac{1}{2}\dot{\mathbf{q}}^\top\mathbf{W}\dot{\mathbf{q}} \\
    \mathbf{W} = m\left(\frac{\partial \mathbf{p}}{\partial \mathbf{q}}\right)^\top \frac{\partial \mathbf{p}}{\partial \mathbf{q}} + \mathbf{\beta}^\top \mathbf{J}\mathbf{\beta}
    \label{eq:mass_inertia_matrix}
\end{align}
\end{subequations}
Where $m$ is the mass, $I$ is the inertia, $W$ is the mass inertia matrix, $p$ is the position vector, $\theta$ is the angle, $v$ is the velocity, $\omega$ is the angular velocity, $h$ is the height, $k$ is the spring constant, and $x$ is the displacement. \newline 
If the object is a point mass, we don't use the inertia term $\frac{1}{2}I\dot{\theta}^2$. \newline
As for equation \eqref{eq:mass_inertia_matrix}, $\mathbf{\omega} = \mathbf{\beta(q)\dot{q}}$ and $\mathbf{J}$ is the inertia matrix. \newline
\subsection{System dynamics}
With this lagrangian, we can describe the system dynamics. Let $\mathbf{Q}$ be the generalized forces, then the equations of motion are:
\begin{subequations}
    \begin{align}
    \mathbf{Q} = \frac{\partial \mathbf{p}}{\partial \mathbf{q}}^\top \mathbf{F} \\
    \frac{d}{dt}\left(\frac{\partial \mathcal{L}}{\partial \dot{\mathbf{q}}}\right) - \frac{\partial \mathcal{L}}{\partial \mathbf{q}} - \frac{\partial \mathbf{c}}{\partial \mathbf{q}}\mathbf{z}= \mathbf{Q} \\
    \frac{d}{dt}\mathbf{W}\mathbf{\dot{q}} - \frac{\partial \mathcal{L}}{\partial \mathbf{q}} = \mathbf{Q} \\
    \mathbf{W}\mathbf{\ddot{q}} + \dot{\mathbf{W}}\mathbf{\dot{q}} - \frac{\partial T}{\partial \mathbf{q}} + \frac{\partial V}{\partial \mathbf{q}} = \mathbf{Q}
    \end{align}    
\end{subequations}
Lastly, we have the \textit{\textbf{GOAT}} matrix, which you get by solving $\frac{d^2}{dt^2}\mathbf{c(q)} = 0$ and using the equation above:
\begin{equation}
    \begin{bmatrix}
        \mathbf{W} & \frac{\partial \mathbf{c}}{\partial \mathbf{q}} \\
        \left(\frac{\partial \mathbf{c}}{\partial \mathbf{q}}\right)^\top & 0
    \end{bmatrix} \begin{bmatrix}
        \mathbf{\ddot{q}} \\
        \mathbf{z}
    \end{bmatrix} = \begin{bmatrix}
        \mathbf{Q} + \frac{\partial T}{\partial \mathbf{q}} - \dot{\mathbf{W}}\dot{\mathbf{q}} - \frac{\partial V}{\partial \mathbf{q}} \\
        - \frac{\partial}{\partial \mathbf{q}} \left(\frac{\partial \mathbf{c}}{\partial \mathbf{q}}\dot{\mathbf{q}}\right) \mathbf{\dot{q}}
    \end{bmatrix}
\end{equation}
\section{Bond Graphs}
test
test2
test3
test(moren din)
\section{Rigid body, i have a rigid body ahhhh}
A \textbf{rigid body}; the distances between any two points within the body remain constant regardless of the external forces and torques applied to it.
\\
\\
\textbf{Degrees of Freedom:} A rigid body in three-dimensional space has six degrees of freedom: three translational (movement along the x, y, and z axes) and three rotational (rotation about the x, y, and z axes).

\subsection{Translational Motion (Newton's Second Law)}
\begin{equation}
    \sum \vec{F} = m \vec{a}_c
\end{equation}

\subsection{Rotational Motion (Euler's Equations)}

The rotational motion of the rigid body is described by Euler's equations, which relate the rate of change of angular momentum to the external moments (torques) acting on the body. These equations are given by:
\begin{equation}
    \sum \vec{T} = \dot{\vec{h}} + \vec{\omega} \times \mathbf{J}\vec{\omega} + r \times m\vec{a}
\end{equation}

where:
\begin{itemize}
    \item $\sum \vec{T}$ is the resultant external moment (torque) acting on the body.
    \item \( \mathbf{J} \) is the inertia tensor of the body.
    \item \( \vec{\omega} \) is the angular velocity of the body.
    \item $\dot{\vec{h}}$ is the time derivative of the angular momentum.
    \item \( \vec{\omega} \times \mathbf{J}\vec{\omega} \) is the term representing the gyroscopic effect.
\end{itemize}

\subsection{Combined Newton-Euler Equations}
The Newton-Euler equations referenced to the center of mass are given by:

\begin{equation}
    \mathbf{F}_{bc} = m \mathbf{a}_c
\end{equation}

\begin{equation}
    \mathbf{T}_{bc} = \mathbf{M}_{b/c} \cdot \boldsymbol{\alpha}_{ib} + \boldsymbol{\omega}_{ib} \times \left( \mathbf{M}_{b/c} \cdot \boldsymbol{\omega}_{ib} \right)
\end{equation}
More complicated equations \\
We use this when the forces and torques are not applied at the center of mass. The equations are given by:
The total torque acting in a arbitrary point P is given by:
\begin{equation}
    \vec{F}_{bo} = m \vec{a}_c
\end{equation}

\begin{equation}
    \mathbf{T}_{p} = \vec{r}_{c/p} \times m \vec{a}_c \cdot \mathbf{M}_{b/c} \cdot \boldsymbol{\alpha}_{ib} + \boldsymbol{\omega}_{ib} \times \left( \mathbf{M}_{b/c} \cdot \boldsymbol{\omega}_{ib} \right)
\end{equation}    
In these equations:
\begin{itemize}
    \item $\vec{r}_{p/c}$ is the vector from the point P to the center of mass.
    \item $\mathbf{F}_{bc}$: Force acting on the body at the center of mass
    \item $m$: Total mass of the body
    \item $\mathbf{a}_c$: Acceleration of the center of mass
    \item $\mathbf{T}_{bc}$: Torque acting on the body at the center of mass
    \item $\mathbf{M}_{b/c}$: Inertia matrix with respect to the center of mass
    \item $\boldsymbol{\alpha}_{ib}$: Angular acceleration of the body relative to the inertial frame
    \item $\boldsymbol{\omega}_{ib}$: Angular velocity of the body relative to the inertial frame
\end{itemize}

\subsection{Inertia matrix}
The inertia matrix \textbf{J} is a 3×3 symmetric matrix that represents the rotational inertia (treghet) of a rigid body. 
\[
\mathbf{J} = \begin{bmatrix}
    J_{xx} & J_{xy} & J_{xz} \\
    J_{yx} & J_{yy} & J_{yz} \\
    J_{zx} & J_{zy} & J_{zz}
\end{bmatrix}
\]
The diagonal elements of the inertia matrix, \( J_{xx} \), \( J_{yy} \), and \( J_{zz} \), are the moments of inertia about the x, y, and z axes, respectively. 
\\
\\
The off-diagonal elements, \( J_{xy} \), \( J_{xz} \), \( J_{yx} \), \( J_{yz} \), \( J_{zx} \), and \( J_{zy} \), are the products of inertia. When the coordinate axes align with the principal axes, these products of inertia are zero, making the inertia matrix diagonal.
\\
\\
If a body has planes of symmetry, the products of inertia with respect to those planes are zero:
\begin{itemize}
    \item One plane of symmetry makes the corresponding products of inertia zero.
    \item Two orthogonal planes of symmetry make more products all products of inertia zero, resulting in a diagonal inertia matrix.
\end{itemize}

\subsection{Finding new COM}

% Generalized formulas for finding a new center of mass and inertia matrix
In order to find the inertia matrix with respect to the new center of mass, we first determine the vector from the original center of mass to the new center of mass. This vector is denoted as $\mathbf{r}_s$.

\begin{equation}
    \mathbf{r}_s = -\frac{m_0}{m_0 + m} \mathbf{r}_0,
\end{equation}
where: 
\begin{itemize}
    \item $m_0$: mass of the added mass
    \item $m$: total mass of the body (including the added mass $m_0$)
    \item $\mathbf{r}_0$: vector from the original center of mass to the added mass $m_0$
    \item $\mathbf{r}_s$: vector from the original center of mass to the new center of mass
\end{itemize}

The new inertia matrix, $\mathbf{M}_0$, with respect to the original center of mass can be calculated as:
\begin{equation}
    \mathbf{M}_0 = \mathbf{M}_{\text{old}} - m_0 \left( \mathbf{r}_0 \times \right) \left( \mathbf{r}_0 \times \right),
\end{equation}
where $\mathbf{M}_{\text{old}}$ is the original inertia matrix.
\\
\\
To find the inertia matrix, $\mathbf{M}_c$, with respect to the new center of mass, we use the parallel axis theorem:
\begin{equation}
    \mathbf{M}_c = \mathbf{M}_0 + (m + m_0) \left( \mathbf{r}_s \times \right) \left( \mathbf{r}_s \times \right),
\end{equation}
\\
\\
Combining the two steps, the complete formula for the new inertia matrix with respect to the new center of mass, $\mathbf{M}_c$, is:
\begin{equation}
    \mathbf{M}_c = \mathbf{M}_{\text{old}} - m_0 \left( \mathbf{r}_0 \times \right) \left( \mathbf{r}_0 \times \right) + (m + m_0) \left( \mathbf{r}_s \times \right) \left( \mathbf{r}_s \times \right).
\end{equation}
\\
\\
In these equations:
\begin{itemize}
    \item $\mathbf{r}_0$: vector from the original center of mass to the added mass $m_0$
    \item $\mathbf{r}_s$: vector from the original center of mass to the new center of mass
    \item $\mathbf{M}_{\text{old}}$: original inertia matrix before adding mass $m_0$
    \item $\mathbf{M}_0$: inertia matrix with respect to the original center of mass, including the added mass $m_0$
    \item $\mathbf{M}_c$: inertia matrix with respect to the new center of mass
\end{itemize}




\subsection{Parallel Axis Theorem}
The Parallel Axis Theorem states that the moment of inertia \(J\) of a rigid body about any axis is given by:

\[
J = J_{\text{cm}} + m d^2
\]

$d$ is the perpendicular distance between the center of mass axis and the new axis.

\subsection{Moments of Inertia}
The moments of inertia describe how mass is distributed relative to an axis:
\begin{align*}
    J_{xx}' &= J_{xx} + m (y_0^2 + z_0^2) \\
    J_{yy}' &= J_{yy} + m (x_0^2 + z_0^2) \\
    J_{zz}' &= J_{zz} + m (x_0^2 + y_0^2)
\end{align*}

\subsection{Products of Inertia}
The products of inertia describe how mass is distributed relative to two different axes:
\begin{align*}
    J_{xy}' &= J_{xy} + m x_0 y_0 \\
    J_{yz}' &= J_{yz} + m y_0 z_0 \\
    J_{zx}' &= J_{zx} + m z_0 x_0
\end{align*}

When the coordinate system is translated, the moments and products of inertia change.

\subsection{Inertia Tensor Transformation (Rotation formula)}

The formula for transforming the inertia tensor due to rotation is:

\[ \mathbf{J}^0 = \mathbf{R}^0_b \mathbf{J} \mathbf{R}^b_0 \]

This formula is used to transform the inertia tensor \(\mathbf{J}\) of a rigid body from its original coordinate frame (frame \(b\)) to a new coordinate frame (frame \(0\)) after a rotation.


\subsection{Simple Physical Pendulum}
The equation of motion for a simple physical pendulum (a uniform rod pivoted at one end) is:
\[
\frac{L^2 m}{3} \ddot{\theta} = -mgL \sin \theta
\]


\subsection{Generalized Rotational Dynamics}
A more general equation for the rotational dynamics of a rigid body is:
\[
\mathbf{J} \dot{\omega} + \vec{r}_{c/a} \times m \vec{a}_c = -mgL \sin \theta \hat{\mathbf{i}}_b
\]
\begin{itemize}
    \item The Parallel Axis Theorem is useful for adjusting the moment of inertia for a shifted axis but does not account for translational acceleration of the pivot.
    \item The generalized rotational dynamics equation includes terms for both rotational inertia and non-inertial effects.
\end{itemize}

The rotational dynamics of a rigid body are described by the equation:

\[
\mathbf{J} \dot{\boldsymbol{\omega}}^b + \boldsymbol{\omega} \times (\mathbf{J} \boldsymbol{\omega}) = \mathbf{T}_c
\]


For rotating bodies about point A:

\[
\dot{h_c} + \vec{r}_{c/a} \times m \vec{a}_c = \vec{T}_A
\]
where 
\[
\dot{h}_c = \mathbf{J} \vec{\dot{\omega}} + \vec{\omega} \times (\mathbf{J} \vec{\omega})
\]

When point \( A \) is fixed in a reference frame of pure rotation, the inertia matrix \( \mathbf{J} \) is updated using the parallel axis theorem:

\[
\mathbf{J}_A = \mathbf{J}_{\text{cm}} + m \mathbf{d} \cdot \mathbf{d}^T - m \|\mathbf{d}\|^2 \mathbf{I}
\]

 $\textbf{d}$ is displacement vector from the center of mass to point \( A \). and $\textbf{I}$ is the identity matrix

The updated rotational dynamics equation is:

\[
\mathbf{J}_A \dot{\boldsymbol{\omega}}^b + \boldsymbol{\omega} \times (\mathbf{J}_A \boldsymbol{\omega}) = \mathbf{T}_A
\]



\section{Generalized Forces}
A generalized force is a force acting along with the generalized coordinate.
\\
\\
In a system described by generalized coordinates $q_i$, the generalized forces $Q_i$ are defined such that they account for the work done by all the actual forces acting on the system.
\subsection{Work in Generalized Coordinates}
The virtual work \( \delta W \) done by generalized forces \( Q_i \) during virtual displacements \( \delta q_i \) is given by:
\[
\delta W =  Q_i \delta q_i
\]


\subsection{Work by Physical Forces}
For a system with physical forces \( \mathbf{F}_j \) acting at points with positions \( \mathbf{p}_j \), the virtual work done by these forces is:
\[
\delta W = \mathbf{F}_j^T \delta \mathbf{p}_j
\]


\subsection{Displacement}
The displacement \( \delta \mathbf{p}_j \) of the point \( \mathbf{p}_j \) can be expressed in terms of the virtual displacements in the generalized coordinates \( \delta q_i \):
\[
\delta \mathbf{p}_j = \sum_{i} \frac{\partial \mathbf{p}_j}{\partial q_i} \delta q_i
\]


\subsection{Generalized Forces in Terms of Physical Forces}
Substituting \( \delta \mathbf{p}_j \) into the expression for virtual work:
\[
\delta W = \mathbf{F}_j^T \left( \sum_{i} \frac{\partial \mathbf{p}_j}{\partial q_i} \delta q_i \right)
\]

Comparing with the generalized virtual work expression, we identify the generalized forces:
\[
Q_{ij} = \mathbf{F}_j^T \frac{\partial \mathbf{p}_j}{\partial q_i}
\]
$i$: index of generalized coordinates 
\\
$j$: index of applied forces

\section{State Space Formulations}

\subsection{Equation of Motion}
The equation of motion derived from Lagrangian mechanics is:
\[
W(q) \ddot{q} + \frac{\partial}{\partial q} (W(q) \dot{q}) \dot{q} - \frac{\partial L}{\partial q} = Q
\]


\subsection{Solving for Acceleration}
Isolate \(\ddot{q}\):
\[
\ddot{q} = W^{-1}(q) \left[ \frac{\partial L}{\partial q} + Q - \frac{\partial}{\partial q} (W(q) \dot{q}) \dot{q} \right]
\]

\subsection{Acceleration Form}
Define the state variables:
\[
x_1 = q, \quad x_2 = \dot{q}
\]
Express the equations of motion in state space form:
\[
\dot{x}_1 = x_2
\]
\[
\dot{x}_2 = W^{-1}(x_1) \left[ \frac{\partial L}{\partial x_1} + Q - \frac{\partial}{\partial x_1} (W(x_1) x_2) x_2 \right]
\]

\section{Nummerical Simulation}
\begin{figure}[H]
    \centering
    \includepdf[width=\linewidth]{figures/sim1.pdf}
\end{figure}
\newpage
\begin{figure}[H]
    \centering
    \includepdf[width=\linewidth]{figures/sim2.pdf}
\end{figure}
\newpage
\subsection{Other notes: }
\subsubsection{Stability}
The region of stability for an RK of order $o$ is defined as: 
\begin{equation}
    S = \left\{\quad \lambda \Delta t \quad \text{s.t.} \quad \left\lvert \sum_{k=0}^{o} \frac{(\lambda \Delta t)^k}{k!}\right\rvert \leq 1 \right\} 
\end{equation}
\subsubsection{Butcher tableau definitions when using stability function}
If you have the tableau: $
    \begin{array}{c|ccc}
    c_1 & a_{11} & \cdots & a_{1s} \\
    \vdots & \vdots & & \vdots \\
    c_s & a_{s1} & \cdots & a_{ss} \\
    \hline
    & b_1 & \cdots & b_s \\
    \end{array}$, then $\mathbf{b} = \begin{bmatrix}b_1 \\ \vdots \\ b_s\end{bmatrix}$, $\mathbf{c} = \begin{bmatrix}c_1 \\ \vdots \\ c_s\end{bmatrix}$ and $\mathbf{A} = \begin{bmatrix}a_{11} & \cdots & a_{1s} \\ \vdots & & \vdots \\ a_{s1} & \cdots & a_{ss}\end{bmatrix}$ \newline
Also noteworthy; the $\mathbf{1}$ vector, is defined as the vector that transforms the term into the desired dimension. From the stability function definition, we have the term $\mathbf{A}-\mathbf{1}\mathbf{b}^\top$. Let's say $\mathbf{b} \in \mathbb{R}^2$ and $\mathbf{A} \in \mathbb{R}^2$, then $\mathbf{1} = \begin{bmatrix}1 \\ 1\end{bmatrix}$.
\subsubsection{IRK vs ERK stability}
Explicit RK methods achieve an order $o = s$ (number of stages) for $s \leq 4$ and then the order “stalls" and does not increase as fast as s. This was bad news for high-order explicit RK methods in terms of efficiency (complexity vs. accuracy). In contrast, implicit RKmethods can achieve $o = 2s$ for any number of stages $s$.
\begin{figure}[h]
    \centering
    \includegraphics[width=\linewidth]{figures/RKStability_Table.pdf}
\end{figure}
\newpage
\subsection{The Forward Euler method}
\begin{subequations}
\begin{align}
    x_{k+1} = x_k + \Delta t \cdot f(x_k)\\
    x_n = n \cdot \Delta t \cdot v \label{eq:Euler}
\end{align}
\end{subequations}

This method is first order, which means global error E is proportional to step size h. 

Stable for: \[\Delta t < - \frac{2}{\lambda}\]

\subsection{Runke gutta 2: Mid Point Method }

Want to evaluate \(\dot{x} = f(x)\) between \(t_k\) and \(t_{k+1}\) (omitting \(u\) and \(t\) for ease of notation):
\begin{equation}
x_{k+1} = x_k + \Delta t f\left( x \left( t_k + \frac{1}{2} \Delta t \right) \right)
\end{equation}

\textbf{Euler Half-Step}
Need to estimate \(x \left( t_k + \frac{1}{2} \Delta t \right)\):
\begin{equation}
x \left( t_k + \frac{1}{2} \Delta t \right) \approx x_k + \frac{1}{2} \Delta t f(x_k)
\end{equation}

The Mid-point Method can also be formulated as:

\begin{align}
k_1 &= f(x_k) \\
k_2 &= f\left(x_k + \frac{1}{2} \Delta t k_1\right) \\
x_{k+1} &= x_k + \Delta t k_2
\end{align}

The Mid-point method is second order.  This means the global error E is proportional to the square of the step size h.
\\
\\
The stability criterion is given by:
\begin{equation}
\left| 1 + \Delta t \lambda + \frac{(\Delta t \lambda)^2}{2} \right| \leq 1
\end{equation}

\textbf{4tho-Order runke gutta }

\begin{align}
k_1 &= f(t_n, x_n) \\
k_2 &= f\left(t_n + \frac{h}{2}, x_n + \frac{h}{2} k_1\right) \\
k_3 &= f\left(t_n + \frac{h}{2}, x_n + \frac{h}{2} k_2\right) \\
k_4 &= f(t_n + h, x_n + h k_3) \\
x_{n+1} &= x_n + \frac{h}{6} \left( k_1 + 2k_2 + 2k_3 + k_4 \right)
\end{align}

The Butcher tableau for the RK4 method is:

\[
\begin{array}{c|cccc}
0 & 0 & 0 & 0 & 0 \\
\frac{1}{2} & \frac{1}{2} & 0 & 0 & 0 \\
\frac{1}{2} & 0 & \frac{1}{2} & 0 & 0 \\
1 & 0 & 0 & 1 & 0 \\
\hline
 & \frac{1}{6} & \frac{1}{3} & \frac{1}{3} & \frac{1}{6} \\
\end{array}
\]
\\
\\
if stiff use implicit runke gutta aahhhh


\subsection{Implecit Runke gutta}
\textbf{A- stability}
\\
\\
A numerical method is said to be A-stable if for any \(\lambda\) with \(\text{Re}(\lambda) \leq 0\), the stability function \(R(z)\) of the method satisfies:

\[
|R(\Delta t \lambda)| \leq 1
\]

for all step sizes \(\Delta t\). where  $|R(\Delta t \lambda)$ is the stabilityf function with the eigenvalues $\lambda$  of the jacobian (usikker om dette), only IRK can be A-stable.
\section{Differential Algebraic Equations (DAE)}
\begin{figure}[H]
    \centering
    \includepdf[width=\linewidth]{figures/dae1.pdf}
\end{figure}
\newpage
\begin{figure}
    \centering
    \includepdf[width=\linewidth]{figures/dae2.pdf}
\end{figure}
You rarely want to reduce to index 0, because the more you differentiate, the more information you lose/the bigger error you introduce.
The consistency conditions are simply the equations that are not differentiated. Your constraints if you are modelling with constrained lagrange for an example. 
\section{Theorems and definitions}
\begin{figure}
    \centering
    \includepdf[width=.9\linewidth]{figures/definitions.pdf}
\end{figure}
%bare et forslag om vi skal ha med dette 
\include{Div matte}


%tanken her er at man har de separat appendix for seg selv printet

\subsection{Newton Euler dynamics and COM}

\begin{figure}[H]
    \centering
    \includegraphics[scale = 0.5]{Skjermbilde 2024-05-23 kl. 18.56.18.png}
    \caption{Caption}
    \label{fig:enter-label}
\end{figure}

(a) Consider the satellite without the added mass. Use the Newton-Euler equations to derive the dynamics of the satellite, i.e., find expressions for $\ddot{\mathbf{r}}_c^i$ and $\dot{\boldsymbol{\omega}}_{ib}^b$.
\\
\\

\textbf{Solution:} The Newton-Euler equations referenced to the center of mass are given by

\begin{equation}
    \mathbf{F}_{bc} = m \mathbf{a}_c
\end{equation}

\begin{equation}
    \mathbf{T}_{bc} = \mathbf{M}_{b/c} \cdot \boldsymbol{\alpha}_{ib} + \boldsymbol{\omega}_{ib} \times \left( \mathbf{M}_{b/c} \cdot \boldsymbol{\omega}_{ib} \right)
\end{equation}

\textcolor{gray}{Expressing the first equation in the inertial frame and the second in body frame gives}

\begin{equation}
    \ddot{\mathbf{r}}_c^i = \frac{\mathbf{F}^i}{m} = -\frac{G m T}{\|\mathbf{r}_c^i\|^2} \frac{\mathbf{r}_c^i}{\|\mathbf{r}_c^i\|}
\end{equation}

\begin{equation}
    \mathbf{M}_c^b \dot{\boldsymbol{\omega}}_{ib}^b = -\left( \boldsymbol{\omega}_{ib}^b \right)^\times \mathbf{M}_c^b \boldsymbol{\omega}_{ib}^b
\end{equation}

\textcolor{gray}{Note that we have used the fact that the gravitational force has zero moment around the center of mass.}
\\
\\
\textcolor{gray}{Further using that $\mathbf{M}_c^b = \frac{1}{6} m l^2 \mathbf{I}$ gives}

\begin{equation}
    \left( \boldsymbol{\omega}_{ib}^b \right)^\times \mathbf{M}_c^b \boldsymbol{\omega}_{ib}^b = \frac{1}{6} m l^2 \left( \boldsymbol{\omega}_{ib}^b \right)^\times \boldsymbol{\omega}_{ib}^b = 0
\end{equation}

\textcolor{gray}{Which gives the simplified expression}

\begin{equation}
    \ddot{\mathbf{r}}_c^i = -\frac{G m T}{\|\mathbf{r}_c^i\|^2} \frac{\mathbf{r}_c^i}{\|\mathbf{r}_c^i\|}
\end{equation}

\begin{equation}
    \dot{\boldsymbol{\omega}}_{ib}^b = 0
\end{equation}
\\
\\
\textbf{b)}
\\
\\
Now consider the added mass (case 2 above). The added mass will shift the center of mass of the system. Calculate the inertia matrix around this new center of mass and find the updated expressions for $\ddot{\mathbf{r}}_c^i$ and $\dot{\boldsymbol{\omega}}_{ib}^b$.
\\
\\
\textbf{Hint:} Use the parallel axis theorem to find the new inertia matrix.
\\
\\
\textbf{Solution:} To find the new inertia matrix with respect to the new center of mass, we first find the inertia matrix with respect to the old center of mass (the center of the cube) with the extra mass added. This is given by
\\
\\
\begin{equation}
    \mathbf{M}_0^b = \frac{1}{6} m l^2 \mathbf{I} - m_0 \left( \mathbf{r}_0^b \times \right) \left( \mathbf{r}_0^b \times \right),
\end{equation}
where $m_0$ is the added mass, and $\mathbf{r}_0$ is the vector from the center of the satellite to the added mass, i.e.,
\begin{equation}
    \mathbf{r}_0^b = \frac{l}{2}
    \begin{bmatrix}
        \pm 1 \\
        \pm 1 \\
        \pm 1
    \end{bmatrix} \quad \text{(the signs depend on the corner)}.
\end{equation}

In order to find the inertia matrix with respect to the center of mass, we have to first find the vector from the center of mass to the center of the cube, $\mathbf{r}_s$. We observe that the addition of a mass $m_0$ at a distance $L$ in one particular direction will shift the center of mass $\frac{m_0}{m + m_0} L$ units in that direction. Hence,
\begin{equation}
    \mathbf{r}_s^b = -\frac{m_0}{m_0 + m} \mathbf{r}_0^b,
\end{equation}
and the parallel axis theorem states
\begin{equation}
    \mathbf{M}_{b/o}^b = \mathbf{M}_{b/c}^b - (m + m_0) \left( \mathbf{r}_s^b \times \right) \left( \mathbf{r}_s^b \times \right) \implies \mathbf{M}_{b/c}^b = \mathbf{M}_{b/o}^b + (m + m_0) \left( \mathbf{r}_s^b \times \right) \left( \mathbf{r}_s^b \times \right).
\end{equation}
Which gives
\begin{equation}
    \mathbf{M}_c^b = \frac{1}{6} m l^2 \mathbf{I} - m_0 \left( \mathbf{r}_0^b \times \right) \left( \mathbf{r}_0^b \times \right) + (m + m_0) \left( \mathbf{r}_s^b \times \right) \left( \mathbf{r}_s^b \times \right).
\end{equation}

The dynamics are now given by
\begin{equation}
    \ddot{\mathbf{r}}_c^i = \frac{\mathbf{F}^i}{m + m_0} = -\frac{G m T}{\|\mathbf{r}_c^i\|^2} \frac{\mathbf{r}_c^i}{\|\mathbf{r}_c^i\|},
\end{equation}

\begin{equation}
    \mathbf{M}_c^b \dot{\boldsymbol{\omega}}_{ib}^b = -\left( \boldsymbol{\omega}_{ib}^b \right)^\times \mathbf{M}_c^b \boldsymbol{\omega}_{ib}^b.
\end{equation}

\textcolor{gray}{Note that we have used the fact that the gravitational force has zero moment around the center of mass. Further using that $\mathbf{M}_c^b = \frac{1}{6} m l^2 \mathbf{I}$ gives}
\begin{equation}
    \left( \boldsymbol{\omega}_{ib}^b \right)^\times \mathbf{M}_c^b \boldsymbol{\omega}_{ib}^b = \frac{1}{6} m l^2 \left( \boldsymbol{\omega}_{ib}^b \right)^\times \boldsymbol{\omega}_{ib}^b = 0,
\end{equation}


\subsection{Kinetic Energy of ball on beam}

\begin{figure}[H]
    \centering
    \includegraphics[scale = 0.4]{Skjermbilde 2024-05-24 kl. 10.57.23.png}
    \caption{Caption}
    \label{fig:enter-label}
\end{figure}

\textbf{(a) The position of the center of mass of the ball as a function of the generalized coordinates is given by:}
\begin{equation}
\mathbf{p} = x \begin{bmatrix} \cos \theta \\ \sin \theta \end{bmatrix} + R \begin{bmatrix} -\sin \theta \\ \cos \theta \end{bmatrix}
\end{equation}

\textbf{(b) The angular velocity of the ball as a function of the generalized coordinates is:}
\begin{equation}
\omega = \dot{\theta} + \frac{\dot{x}}{R}
\end{equation}

\textbf{(c) The kinetic energy of the ball, which experiences both translation and rotation, is given by:}

\textit{Translational Kinetic Energy}
\begin{equation}
T_T(q, \dot{q}) = \frac{1}{2} \dot{\mathbf{q}}^T M_T(q) \dot{\mathbf{q}}, \quad M_T(q) = M \frac{\partial \mathbf{p}}{\partial \mathbf{q}}^T \frac{\partial \mathbf{p}}{\partial \mathbf{q}}
\end{equation}

\textit{Rotational Kinetic Energy}
\begin{equation}
T_R(q, \dot{q}) = \frac{1}{2} I_{\text{ball}} \omega^2, \quad I_{\text{ball}} = \frac{2}{5} M R^2
\end{equation}

\textit{Total Kinetic Energy}
\begin{equation}
T(q, \dot{q}) = T_T + T_R
\end{equation}

\textbf{(d) The kinetic energy of the beam, which only rotates, is given by:}
\begin{equation}
T_r(q, \dot{q}) = \frac{1}{2} J \dot{\theta}^2
\end{equation}

\textbf{Overall Kinetic Energy}
\begin{equation}
T(q, \dot{q}) = T_T + T_R + T_r
\end{equation}

\textbf{Potential Energy}
\begin{equation}
V(q) = Mg p(2) = Mg (x \sin \theta + R \cos \theta)
\end{equation}

\textbf{Generalized Forces}
\begin{equation}
\mathbf{Q} = \begin{bmatrix} 0 \\ T \end{bmatrix}
\end{equation}


\subsection{Pendelum Lagrange example}

\begin{figure}[H]
    \centering
    \includegraphics[scale = 0.4]{Skjermbilde 2024-05-24 kl. 14.32.36.png}
    \caption{Caption}
    \label{fig:enter-label}
\end{figure}

\textbf{a)} Select a set of generalized coordinates. 
\\
\\
solution: \[vec{q} = [z, \theta]^T\]
\\
\\
\textbf{b)} Find the kinetic energy of the system and express it in terms of the generalized coordinates (and their time derivatives).
\\
\\
\textbf{Solution:} To find the kinetic energy as a function of the generalized coordinates (and their time derivatives) we need to find the velocity components of the masses. We will express the coordinates of the masses in terms of the generalized coordinates and then differentiate. For \(m_1\), only the \(z\)-coordinate is relevant. As seen in the figure, the variable \(z\) is the \(z\)-coordinate of \(m_1\).
\\
\\
Mass \(m_2\) has the following coordinates:
\[
y_2 = L \sin \theta
\]
\[
z_2 = z - L \cos \theta
\]
\\
\\
Differentiating our 3 coordinates, gives
\[
\dot{z}_1 = \dot{z}
\]
\[
\dot{y}_2 = L \cos \theta \dot{\theta}
\]
\[
\dot{z}_2 = \dot{z} + L \sin \theta \dot{\theta}
\]
\\
\\
The kinetic energy can then be found as
\[
T = \frac{1}{2} m_1 \dot{z}_1^2 + \frac{1}{2} m_2 \dot{y}_2^2 + \frac{1}{2} m_2 \dot{z}_2^2
\]
\[
= \frac{1}{2} (m_1 + m_2) \dot{z}^2 + \frac{1}{2} L^2 m_2 \dot{\theta}^2 + L m_2 \dot{\theta} \dot{z} \sin \theta
\]
\\
\\
\textbf{c)} Find the potential energy of the system and express it in terms of the generalized coordinates.
\\
\\
\textbf{Solution:} The potential energy has three contributions. The first and second contribution is related to the energy storage from elevating the masses in the gravitational field, and the third contribution comes from compressing or stretching the spring. The following expression represents these contributions:
\\
\\
\[
V = z m_1 g + (z - L \cos \theta) m_2 g + \frac{1}{2} kz^2
\]
\\
\\
\textbf{d:} Find equations of motion for the system.
\\
\\
\textbf{Solution:} We start by finding the partial derivative of the kinetic energy with respect to each of the generalized rates:
\[
\frac{\partial T}{\partial \dot{z}} = (m_1 + m_2) \dot{z} + L m_2 \dot{\theta} \sin \theta
\]
\[
\frac{\partial T}{\partial \dot{\theta}} = L^2 m_2 \dot{\theta} + L m_2 \dot{z} \sin \theta
\]

We then find the time derivatives of the above expressions:
\[
\frac{d}{dt} \left( \frac{\partial T}{\partial \dot{z}} \right) = (m_1 + m_2) \ddot{z} + L m_2 \ddot{\theta} \sin \theta + L m_2 \dot{\theta}^2 \cos \theta
\]
\[
\frac{d}{dt} \left( \frac{\partial T}{\partial \dot{\theta}} \right) = L^2 m_2 \ddot{\theta} + L m_2 \ddot{z} \sin \theta + L m_2 \dot{z} \dot{\theta} \cos \theta
\]

We proceed by finding the partial derivatives of the kinetic energy with respect to the generalized coordinates:
\[
\frac{\partial T}{\partial z} = 0
\]
\[
\frac{\partial T}{\partial \theta} = L m_2 \dot{\theta} \dot{z} \cos \theta
\]

and the partial derivatives of the potential energy with respect to the generalized coordinates
\[
\frac{\partial V}{\partial z} = (m_1 + m_2) g + k z
\]
\[
\frac{\partial V}{\partial \theta} = L m_2 g \sin \theta
\]

Finally we use the Lagrange formula
\[
\frac{d}{dt} \left( \frac{\partial T}{\partial \dot{q}_i} \right) - \frac{\partial T}{\partial q_i} + \frac{\partial V}{\partial q_i} = 0
\]

Filling in the expressions found above gives
\[
\frac{d}{dt} \left( \frac{\partial T}{\partial \dot{z}} \right) - \left( \frac{\partial T}{\partial z} - \frac{\partial V}{\partial z} \right) = 0
\]
\[
\frac{d}{dt} \left( \frac{\partial T}{\partial \dot{\theta}} \right) - \left( \frac{\partial T}{\partial \theta} - \frac{\partial V}{\partial \theta} \right) = 0
\]
\[
L^2 m_2 \ddot{\theta} + L m_2 \ddot{z} \sin \theta + L m_2 \dot{z} \dot{\theta} \cos \theta + L m_2 g \sin \theta = 0
\]

% References
%\newpage
%\addcontentsline{toc}{section}{References}
%\bibliographystyle{unsrt}
%\bibliography{bibliography.bib}
%\label{sec:bibliography}

\end{document}
