\section{Rigid body, i have a rigid body ahhhh}
A \textbf{rigid body}; the distances between any two points within the body remain constant regardless of the external forces and torques applied to it.
\\
\\

\textbf{Degrees of Freedom:} A rigid body in three-dimensional space has six degrees of freedom: three translational (movement along the x, y, and z axes) and three rotational (rotation about the x, y, and z axes).
\\
\\
\subsection{Equation of motion}

Rigid body dynamics is governed by Newton's laws of motion and Euler's rotation equations:

\subsection{Newton’s Second Law for Translation}

\begin{equation}
\Sigma\mathbf{F} = m \mathbf{a}
\end{equation}
where 
\(\mathbf{F}\) is the net force acting on the body, 
\(m\) is the mass of the body, and 
\(\mathbf{a}\) is the linear acceleration of the center of mass.

\subsection{Newton’s Second Law for Rotation (Euler’s Equations)}

\begin{equation}
\Sigma\mathbf{T} = \mathbf{J} \boldsymbol{\alpha} + \boldsymbol{\omega} \times (\mathbf{J} \boldsymbol{\omega})
\end{equation}

where 
\(\mathbf{T}\) is the net torque, 
\(\mathbf{J}\) is the inertia matrix, 
\(\boldsymbol{\alpha}\) is the angular acceleration (\(\boldsymbol{\Dot{\omega}}\)), and 
\(\boldsymbol{\omega}\) is the angular velocity.
\\
\\




The inertia matrix \textbf{J} is a 3×3 symmetric matrix that represents the rotational inertia (treghet) of a rigid body. It is defined with respect to a coordinate system whose origin is at a point (often the center of mass) and aligned with the principal axes of the body.
\[
\mathbf{J} = \begin{bmatrix}
    J_{xx} & J_{xy} & J_{xz} \\
    J_{yx} & J_{yy} & J_{yz} \\
    J_{zx} & J_{zy} & J_{zz}
\end{bmatrix}
\]
