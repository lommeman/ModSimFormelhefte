\section{Rigid body, i have a rigid body ahhhh}
A \textbf{rigid body}; the distances between any two points within the body remain constant regardless of the external forces and torques applied to it.
\\
\\
\textbf{Degrees of Freedom:} A rigid body in three-dimensional space has six degrees of freedom: three translational (movement along the x, y, and z axes) and three rotational (rotation about the x, y, and z axes).

\subsection{Translational Motion (Newton's Second Law)}
\[
\vec{F}_{\text{ext}} = m \vec{a}_{\text{cm}}
\]

\subsection{Rotational Motion (Euler's Equations)}

The rotational motion of the rigid body is described by Euler's equations, which relate the rate of change of angular momentum to the external moments (torques) acting on the body. These equations are given by:

\[
\vec{T}_{\text{ext}} = \frac{d \vec{H}}{dt} + \vec{\omega} \times \vec{H}
\]

where:
\begin{itemize}
    \item \( \vec{T}_{\text{ext}} \) is the resultant external moment (torque) acting on the body.
    \item \( \vec{H} = \mathbf{J} \vec{\omega} \) is the angular momentum of the body.
    \item \( \mathbf{J} \) is the inertia tensor of the body.
    \item \( \vec{\omega} \) is the angular velocity of the body.
    \item \( \frac{d \vec{H}}{dt} \) is the time derivative of the angular momentum.
    \item \( \vec{\omega} \times \vec{H} \) is the term representing the gyroscopic effect.
\end{itemize}

\subsection{Combined Newton-Euler Equations}

Combining the translational and rotational equations, the complete Newton-Euler equations for the motion of a rigid body are:

\[
\vec{F}_{\text{ext}} = m \vec{a}_{\text{cm}}
\]

\begin{equation}
\Sigma\vec{T} = \mathbf{J} \vec{\alpha} + \vec{\omega} \times (\mathbf{J} \vec{\omega})
\end{equation}

This is also called angular momentum, denoted \( \dot{h_c}\) (?).

\begin{itemize}
    \item \textbf{Angular Momentum} (\( \vec{H} \)): The product of the inertia tensor and the angular velocity of the body, representing the rotational state of the body.
    \item \textbf{Gyroscopic Effect} (\( \vec{\omega} \times \vec{H} \)): A term representing the precessional motion caused by the body's rotation.
\end{itemize}
\\
\\
\subsection{Inertia matrix}
The inertia matrix \textbf{J} is a 3×3 symmetric matrix that represents the rotational inertia (treghet) of a rigid body. 
\[
\mathbf{J} = \begin{bmatrix}
    J_{xx} & J_{xy} & J_{xz} \\
    J_{yx} & J_{yy} & J_{yz} \\
    J_{zx} & J_{zy} & J_{zz}
\end{bmatrix}
\]
The diagonal elements of the inertia matrix, \( J_{xx} \), \( J_{yy} \), and \( J_{zz} \), are the moments of inertia about the x, y, and z axes, respectively. 
\\
\\
The off-diagonal elements, \( J_{xy} \), \( J_{xz} \), \( J_{yx} \), \( J_{yz} \), \( J_{zx} \), and \( J_{zy} \), are the products of inertia. When the coordinate axes align with the principal axes, these products of inertia are zero, making the inertia matrix diagonal.
\\
\\
If a body has planes of symmetry, the products of inertia with respect to those planes are zero:
\begin{itemize}
    \item One plane of symmetry makes the corresponding products of inertia zero.
    \item Two orthogonal planes of symmetry make more products all products of inertia zero, resulting in a diagonal inertia matrix.
\end{itemize}

\subsection{Parallel Axis Theorem}
The Parallel Axis Theorem states that the moment of inertia \(J\) of a rigid body about any axis is given by:

\[
J = J_{\text{cm}} + m d^2
\]

$d$ is the perpendicular distance between the center of mass axis and the new axis.

\subsection{Moments of Inertia}
The moments of inertia describe how mass is distributed relative to an axis:
\begin{align*}
    J_{xx}' &= J_{xx} + m (y_0^2 + z_0^2) \\
    J_{yy}' &= J_{yy} + m (x_0^2 + z_0^2) \\
    J_{zz}' &= J_{zz} + m (x_0^2 + y_0^2)
\end{align*}

\subsection{Products of Inertia}
The products of inertia describe how mass is distributed relative to two different axes:
\begin{align*}
    J_{xy}' &= J_{xy} + m x_0 y_0 \\
    J_{yz}' &= J_{yz} + m y_0 z_0 \\
    J_{zx}' &= J_{zx} + m z_0 x_0
\end{align*}

When the coordinate system is translated, the moments and products of inertia change.

\subsection{Inertia Tensor Transformation (Rotation formula)}

The formula for transforming the inertia tensor due to rotation is:

\[ \mathbf{J}^0 = \mathbf{R}^0_b \mathbf{J} \mathbf{R}^b_0 \]

This formula is used to transform the inertia tensor \(\mathbf{J}\) of a rigid body from its original coordinate frame (frame \(b\)) to a new coordinate frame (frame \(0\)) after a rotation.


\subsection{Simple Physical Pendulum}
The equation of motion for a simple physical pendulum (a uniform rod pivoted at one end) is:
\[
\frac{L^2 m}{3} \ddot{\theta} = -mgL \sin \theta
\]


\subsection{Generalized Rotational Dynamics}
A more general equation for the rotational dynamics of a rigid body is:
\[
\mathbf{J} \dot{\omega} + \vec{r}_{c/a} \times m \vec{a}_c = -mgL \sin \theta \hat{\mathbf{i}}_b
\]
\begin{itemize}
    \item The Parallel Axis Theorem is useful for adjusting the moment of inertia for a shifted axis but does not account for translational acceleration of the pivot.
    \item The generalized rotational dynamics equation includes terms for both rotational inertia and non-inertial effects.
\end{itemize}

The rotational dynamics of a rigid body are described by the equation:

\[
\mathbf{J} \dot{\boldsymbol{\omega}}^b + \boldsymbol{\omega} \times (\mathbf{J} \boldsymbol{\omega}) = \mathbf{T}_c
\]


For rotating bodies about point A:

\[
\dot{h_c} + \vec{r}_{c/a} \times m \vec{a}_c = \vec{T}_A
\]
where 
\[
\dot{h}_c = \mathbf{J} \vec{\omega} + \vec{\omega} \times (\mathbf{J} \vec{\omega})
\]

When point \( A \) is fixed in a reference frame of pure rotation, the inertia matrix \( \mathbf{J} \) is updated using the parallel axis theorem:

\[
\mathbf{J}_A = \mathbf{J}_{\text{cm}} + m \mathbf{d} \cdot \mathbf{d}^T - m \|\mathbf{d}\|^2 \mathbf{I}
\]

 $\textbf{d}$ is displacement vector from the center of mass to point \( A \). and $\textbf{I}$ is the identity matrix

The updated rotational dynamics equation is:

\[
\mathbf{J}_A \dot{\boldsymbol{\omega}}^b + \boldsymbol{\omega} \times (\mathbf{J}_A \boldsymbol{\omega}) = \mathbf{T}_A
\]


